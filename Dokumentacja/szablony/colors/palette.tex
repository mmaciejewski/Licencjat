\documentclass[12pt,a4paper]{article}

\RequirePackage{polyglossia}
\setdefaultlanguage{polish}

\usepackage{multicol}
\pagestyle{empty}

\setlength{\oddsidemargin}{0pt}
\setlength{\textwidth}{16cm}
\setlength{\textheight}{22cm}
\setlength{\parindent}{0pt}
\setlength{\parskip}{0pt}

\usepackage[dvipsnames]{color}
\definecolor{sa}{named}{RubineRed}
\definecolor{sb}{named}{JungleGreen}
\definecolor{sc}{named}{Bittersweet}
\definecolor{wa}{named}{RubineRed}
\definecolor{wb}{named}{JungleGreen}
\definecolor{wc}{named}{Bittersweet}
\definecolor{wd}{named}{MidnightBlue}

%% \usepackage{color}
%% \definecolor{sa}{cmyk}{0,1,0.13,0}
%% \definecolor{sb}{cmyk}{0.99,0,0.52,0}
%% \definecolor{sc}{cmyk}{0,0.75,1,0.24}
%% \definecolor{wa}{cmyk}{0,1,0.13,0}
%% \definecolor{wb}{cmyk}{0.99,0,0.52,0}
%% \definecolor{wc}{cmyk}{0,0.75,1,0.24}
%% \definecolor{wd}{cmyk}{0.98,0.13,0,0.43}

\begin{document}
\renewcommand*{\DefineNamedColor}[4]{%
  \textcolor[named]{#2}{\rule{7mm}{7mm}}\quad
  \texttt{#2}\strut\\}

\begin{center}
  \Large Named colors in \texttt{dvipsnam.def}
\end{center}

\begin{multicols}{3}
\input{dvipsnam.def}
\end{multicols}

\newpage

Streszczenie:

\begin{description}
\item[sa] \textcolor{sa}{powinno zawierać omówienie głównych tez pracy magisterskiej, celów jakie autor sobie postawił}
\item[sb] \textcolor{sb}{powinno zawierać informację czy udało się je zrealizować}
\item[sc] \textcolor{sc}{należy także napisać jakimi metodami, technologiami się posłużono i jakie to przyniosło efekty}
\end{description}


Wstęp:

\begin{description}
\item[wa] \textcolor{wa}{jak nazwa wskazuje, ma wprowadzać w obszar problemowy pracy}
\item[wb] \textcolor{wb}{powinno przedstawiać ogólne uwarunkowania problemu oraz opisać go w kontekście}
\item[wc] \textcolor{wc}{powinno zawierać powód dlaczego poruszyło się taki temat}
\item[wd] \textcolor{wd}{należy odnieść się do dorobku innych}
\end{description}

\end{document}
