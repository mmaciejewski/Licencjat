\documentclass[brudnopis]{xmgr}
% Jeśli nowe rozdziały mają się zaczynać na stronach
% nieparzystych:
%\documentclass[openright]{xmgr}

%\defaultfontfeatures{Scale=MatchLowercase}
%\setmainfont[Numbers=OldStyle,Ligatures=TeX]{Minion Pro}
%\setsansfont[Numbers=OldStyle,Ligatures=TeX]{Myriad Pro}
% for fontspec version < 2.0
\setmainfont[Numbers=OldStyle,Mapping=tex-text]{Minion Pro}
\setsansfont[Numbers=OldStyle,Mapping=tex-text]{Myriad Pro}
%\setmonofont[Scale=0.75]{Monaco}

% Opcjonalnie identyfikator dokumentu
% drukowany tylko z włączoną opcją 'brudnopis':
\wersja   {wersja wstępna [\ymdtoday]}

\author   {Czerwińska Agnieszka}
\nralbumu {206\,314}
\email    {czewinska.agnieszka.lucja@gmail.com}

\author   {Maciejewski Michał}
\nralbumu {206\,316}
\email    {michal.maciejewski92@gmail.coml}

\author   {Miszczykowski Mariusz}
\nralbumu {000\,000}
\email {000@gmail.com}

\title    {Tworzenie gier na przykładzie silników Unity i Unreal Engine}
\date     {2016}
\miejsce  {Gdańsk}

\opiekun  {dr W. Bzyl}

% dodatkowe polecenia
%\renewcommand{\filename}[1]{\texttt{#1}}
%\definecolor{stress}{cmyk}{0,1,0.13,0} % RubineRed
%\definecolor{topic}{cmyk}{0.98,0.13,0,0.43} % MidnightBlue

\begin{document}

% streszczenie
\begin{abstract}
\end{abstract}

% słowa kluczowe
\keywords{gry konsolowe, 
silnik gier, 
Unity, 
Unreal Engine, 
animacja,
fizyka,
skrypty,
interfejs użytkownika,
scena,
obiekt,
platforma,
optymalizacja,
particle,
audio,
3D,
architektura,
oswietlenie,
komponent,
drag and drop}

% tytuł i spis treści
\maketitle

% wstęp
\introduction

Jeszcze 20 lat temu tworzenie gier było nie lada wyzwaniem, wymagało ogromnej cierpliwości, sporej wiedzy z zakresu fizyki, informatyki i matematyki. Dzisiaj grę stworzyć może nawet przeciętny nastolatek, dlatego liczba gier wychodzących na rynek jest tak duża, że nie sposób śledzić pomniejsze tytuły.

Odopowiedzią są silniki gier - zaawansowane oprogramowanie tworzone przez firmy specjalnie na potrzeby nowych gier. Powołanie nowej gry do zycia wymaga ogromnych nakładow czasowych i finansowych. "Jak bedziemy generować grafikę? Czy w naszej grze będzie realistyczna fizyka? A dynamiczne światło? Co z dzwiękiem? Jaki język programowania wybrać? A i trzeba jeszcze tę grę zoptymalizować pod konkretną platformę!". Na te i wiele innych pytań trzeba odpowiedzieć za każdym razem tworząc nową grę. Doświadczeni twórcy wiedzą jednak, że to własnie odpowiedzi na te pytania generują największe koszty. Aby uniknąć odkrywania koła na nowo, firmy często tworzą własne silniki, bądź wykupują licencje na inne. Współcześnie, większość nowych tytułów robiona jest na już gotowych silnikach. Tylko mały procent doczeka się własnego, wartego niejednokrotnie więcej niż sama gra. Na nasze szczęście dzisiaj każdy z nas może skorzystać z możliwości takiego silnika. O ile kiedyś były one zarezerwowane tylko dla wąskiej grupy twórców, szybko odkryto, że na sprzedaży i udostępnianiu silników gier można także zarobić spore pieniądze. Koszty licencji popularnych płatnych silników takich jak Frostbite 3 lub Cry Engine 3 mogą sięgać milionowych kwot.

Mimo, że na rynku istnieje wiele innych darmowych silnikow, w tej pracy skupimy sie na dwóch obecnie najpopularniejszych darmowych silnikach: Unreal Engine udostępniane przez firmę Epic Games, oraz Unity 3D udostępniane przez firmę Unity Technologies.

Przyjrzymy się krótkiej historii silników gier oraz porównamy oba oprogramowania. Dowiemy sie czym jest GDD, następnie przyjrzymy się ich cechom i opiszemy proces tworzenia gier w każdym z nich. Odkryjemy najczęstsze bugi w grach i jak z nimi walczyć, a w podsumowaniu opiszemy krótko: stopień trudnosci w użytkowaniu, czas potrzebny do stworzenia gry i zastanowimy sie nad przyszłością obu silników.

\chapter{Krótka historia silników do tworzenia gier}

Historia gier sięga roku 1942, kiedy to dwoje amerykanów Thomasa A. Goldsmith Jr. oraz Estle Ray, tworzy prototyp gry komputerowej : Symulator pocisku rakietowego. Kolejne lata przynoszą nowe technologie. Rok 1952 - pierwsza gra korzystająca z prototypu monitora. 1958 - pierwsze kontrolery. 1961 - pierwsza gra z wektorową grafiką.

Jednak to dopiero rok 1972 można uznać za początek nowej branży, kiedy powstaje Pong - pierwsza gra odnosząca komercyjny sukces. Udowadnia ona, że na grach można zarobić pieniądze. Popyt okazał się ogromny, zaś rynek wyjątkowo mały.

Nie można zapominać, że były to czasy gdy języki programowania znane nam dzisiaj nie istniały. Były to pierwsze lata języka C. Najpopularniejszy powszechnie język do tworzenia silników gier - C++ - nie istniał wogóle. Większość gier była tworzona od podstaw: budowano odpowiednie maszyny. Tak popularne w tamych czasach automaty nie byly niczym wiecej niz ladnie obudowanymi gotowymi urzadzeniami. Każdy był tworzony pod konkretną grę - z zamkniętą architekturą i bez możliwości wprowadzania zmian.

Dopiero rozwój komputerów osobistych pozwolił na większą swobodę oraz pisanie wielu programów pod jedno urządzenie. Nie zmieniało to faktu, że ze względu na różnorodną architekturę, kod napisany raz pod konkretna gre, po jej wydaniu stawal sie bezuzyteczny. Dodatkowo, wczesny sprzet komputerowy cechowal sie niewielka wydajnoscia, totez aby uzyskac jak najlepsza wydajnosc, pisano kod bezposrednio pod dany sprzet.

\chapter{Czym jest GDD?}

Tekst

\chapter{Przedstawienie Unity i Unreal Engine}

Tekst

\chapter{Podstawy tworzenia gier}

Tekst

\section{Unity}

Tekst

\section{Unreal Engine}

Tekst

\chapter{Prosta gra krok po kroku - Unity}

Tekst

\chapter{Prosta gra krok po kroku - Unreal Engine}

Tekst

\chapter{Testowanie gier}

Tekst

\chapter{Podsumowanie procesu tworzenia gier dla obu silników}

Tekst

% zakończenie
\summary
Tekst

% załączniki (opcjonalnie):
\appendix
\chapter{Tytuł załącznika jeden}

Treść załącznika jeden.

\chapter{Tytuł załącznika dwa}

Treść załącznika dwa.

% literatura (obowiązkowo):
\bibliographystyle{unsrt}
\bibliography{xml}

% spis tabel (jeżeli jest potrzebny):
\listoftables

% spis rysunków (jeżeli jest potrzebny):
\listoffigures

\oswiadczenie

\end{document}